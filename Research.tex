\documentclass{article}

\usepackage{amsmath}

%Probability Commands
\newcommand \E[1] {\mathrm E \left[#1\right]} %Expected Value
\newcommand \Var[1] {\mathrm {Var} \left(#1\right)} %Variance
\newcommand \Cov[2] {\mathrm {Cov} (#1, #2)} %Covariance

%Matrix Commands
\newcommand \inv [1] {{#1}^{-1}} %Inverse

\newcommand \vitr [2] {#1^{(#2)}} %An element of a sequence of vectors

\begin{document}

\section{New Random Edge Weights at Each Iteration}

Define the PageRank iteration $\vitr x {k+1} =\alpha (P+Q_k)\vitr x k
+ (1-\alpha)v$, where each $Q_k$ is independently drawn from the
probability distribution function $f$, and the other variables are
defined as usual. We require that $f$ be defined such that $P+Q_k$ is
always a valid PageRank matrix. For example, $P$ and $P+Q_k$ must be
column stochastic, so the elements of any column of $Q_k$ must sum to
zero. Furthermore, without loss of generality, assume that $\E f =
0$. (If $\E f$ was not zero, we could redefine $P$ to be $P+\E f$ and
$f$ to be $f-\E f$.)

By the linearity of expectation, $\E {\vitr x {k+1}} = \alpha P
\E{\vitr x k} + (1-\alpha)v$, and of course $\E {\vitr x 0} = \vitr x
0$, so computing $\E {\vitr x {k+1}}$ is equivalent to a traditional
PageRank problem without randomness. Determining the likely accuracy
of this iteration is more complicated. (Note: To ease notation, I use
$y = \vitr x {k+1}$ and $x = \vitr x k$ for the rest of this section.)
My first attempt was to compute $\Var{\| y\|_2}$, as follows:

\begin{align*}
\Var{\|y\|} &= \E{\|y\|^2}-\E{\|y\|}^2\\ &= \E{\sum_{i=1}^n
  y_i^2}-\E{\|y\|}^2\\ &= \left(\sum_{i=1}^n
\E{y_i^2}\right)-\E{\|y\|}^2
\end{align*}

I don't know how to compute the last term, but knowing $\E{y_i^2}$
would allow us to compute $\Var{y_i}$, since
$\Var{y_i}=\E{y_i^2}-\E{y_i}^2$, and we know how to compute
$\E{y_i}$. More generally, I derived an algorithm for computing
$\E{yy^T}$ given $\E{xx^T}$. (This allows us to recursively compute
$\E{yy^T}$.) We can read the expected values of the squares of the
elements of $y$ off the diagonal of this matrix. Furthermore,
$\E{yy^T}-\E{y}\E{y}^T$ is the covariance matrix of the elements of
$y$, since
$\left(\E{yy^T}-\E{y}\E{y}^T\right)_{i,j}=\E{y_iy_j}-\E{y_i}\E{y_j}=\Cov{y_i}{y_j}$.

Suppose we're given $Q$, and let $A=P+Q$, so $y=\alpha Ax +
(1-\alpha)v$. Then,

\begin{align*}
\E{y y^T \mid Q} &= \E{\left(\alpha Ax+(1-\alpha)v\right) \left(\alpha
  x^T A^T+(1-\alpha)v^T\right) \mid Q} \\ &= \alpha^2 A\E{x x^T}A^T +
\alpha(1-\alpha)(A \E x v^T + v {\E x}^T A^T )+ (1-\alpha)^2 vv^T
\end{align*}

Since $\E Q = 0$ and $\E A = \E P + \E Q = P$, we have the following:

\begin{align*}
\E{yy^T} = \alpha^2 P\E{xx^T}P^T + \alpha^2 \E{Q\E{xx^T}Q^T} +
\alpha(1-\alpha)(P \E x v^T + v {\E x}^T P^T )+ (1-\alpha)^2 vv^T
\end{align*}

Since we know $\E x$ and $\E{xx^T}$, the only term we don't know is
$\E{Q\E{xx^T}Q^T}$. For all integers $q,r,s,t$ from $1$ to $n$, let
$c_{q,r,s,t} = \Cov{f_{q,r}}{f_{s,t}} =
\E{f_{q,r}f_{s,t}}-\E{f_{q,r}}\E{f_{s,t}}=\E{f_{q,r}f_{s,t}}$, the
covariance of two elements of the output of $f$. (These values can be
precomputed before any iterations of this algorithm.) Then
$(Q\E{xx^T}Q^T)_{q,s} = Q_{q,:}\E{xx^T}(Q_{s,:})^T = \sum_{r,t=1}^n
Q_{q,r}\E{xx_T}_{r,t}Q_{s,t}$, and its expected value is
$\sum_{r,t=1}^n c_{q,r,s,t}\E{xx^T}_{r,t}$. The entire matrix
$\E{Q\E{xx^T}Q^T}$ can be computed by iterating over $r$ and $t$ from
$1$ to $n$.

Some remaining problems include finding a closed form for this
algorithm, and/or an approximation algorithm that makes it easier to
understand how the variance of $f$ results in variance in the PageRank
vector, especially as $k$ becomes large. I decided to hold off on that
for now in case what I've done so far is on the wrong track.

\section{Approximations Regarding Random Edge Weights in a Direct Solution}

Suppose we have a PageRank Problem $Mx=b$, but now we add randomness,
so we try to solve $(M-E)x_e = b$, where $E$ is random and has
expected value 0. Assume that $E$ is small enough so that $(M-E)^{-1}
= \sum_{k=0}^\infty \inv M (E\inv M)^k$. If $E$ is guaranteed to be
sufficiently small, then $\inv{(M-E)} \approx \inv M + \inv M E \inv
M$. ($\inv M E\inv M E\inv M$ is a good estimate of the error of this
first order approximation, for sufficiently small $E$.) This allows us
to approximate the effect of randomness on the solution:

\begin{align*}
x_e - x &= \inv{(M-E)}b - x\\ &= \inv{(M-E)}Mx - x\\ &=
\left(\inv{(M-E)}M-I\right)x\\ &\approx \left((\inv M + \inv M E \inv
M)M-I\right)x\\ &= (I + \inv M E - I)x\\ &= \inv M E x
\end{align*}

It follows that $\E{x_e-x}\approx\inv M \E E x=0$, and also,
$\E{\inv{(M-E)}}\approx \inv M + \inv M \E E \inv M = \inv
M$. Regardless of the approximation used, knowing $\E{(M-E)^{-1}}$
allows us to find the expected solution of the PageRank problem, since
$\E{(M-E)^{-1}b}=\E{(M-E)^{-1}}b$. If instead we want a second order
approximation of $\inv{(M-E)}$, we can compute its expected value as
follows:

\begin{align*}
\E{\inv{(M-E)}} &\approx \E{\inv M + \inv M E \inv M + \inv M E \inv M
  E \inv M} \\ &= \inv M + \inv M \E{E\inv M E} \inv M
\end{align*}

$\E{E\inv M E}$ depends on the covariances of the elements of $E$, and
can be computed in a manner similar to that in the last paragraph of
the previous section. Similarly, a third-order approximation also
depends on $\inv M \E{E \inv M E \inv M E} \inv M$, which depends on
the expected values of products of three elements of $E$. The same
pattern holds for higher-order approximations.

The variance of the first-order approximation of $\inv{(M-E)}$ is
$\Var{\inv M + \inv M E \inv M} = \Var{\inv M E \inv M} = \E{({\inv M
    E \inv M})^2}-\E{\inv M E \inv M}^2 = \E{\inv M E M^{-2} E \inv
  M}-0^2 = \inv M \E{EM^{-2}E}\inv M$, which depends on the
covariances of the elements of $E$. The variance of the second-order
approximation can be computed as follows:

\begin{align*}
\Var{\cdot} &= \Var{\inv M + \inv M E \inv M + \inv M E \inv M E \inv
  M} \\ &= \Var{\inv M E \inv M + \inv M E \inv M E \inv M}\\ &=
M^{-2} \Var{E+E\inv M E} M^{-2}\\ &= M^{-2} \left(\E{(E+E\inv M
  E)^2}-\E{E+E\inv ME}^2\right)M^{-2}\\ &= M^{-2}
\left(\E{E^2}+\E{E^2\inv M E}+\E{E\inv ME^2}+\E{E\inv M E^2\inv M E} -
\E{E\inv M E}^2\right) M^{-2}
\end{align*}

If the $k$th order approximation of $(M-E)^{-1}$ is $\sum_{i=0}^k
(\inv M E)^i \inv M$, then its error is $\sum_{i=k+1}^\infty (\inv M
E)^i \inv M = (\inv M E)^{k+1} \sum_{i=0}^\infty (\inv M E)^i \inv M =
(\inv M E)^{k+1}(M-E)^{-1}$. We can bound the relative error of this
approximation as follows:

%The error in the first-order approximation is $\sum_{i=2}^\infty
%(\inv M E)^i \inv M = (\inv M E)^2\sum_{i=0}^\infty (\inv M E)^i \inv
%M = \inv M E \inv M E \inv{(M-E)}$. We can bound the relative error
%of this approximation as follows:

\begin{align*}
\frac{\|(\inv M E)^{k+1}\inv{(M-E)}\|}{\|\inv{(M-E)}\|} &\leq
\frac{(\|\inv M\|\| E\|)^{k+1}\|(M-E)^{-1}\|}{\|(M-E)^{-1}\|} \\ &=
(\|\inv M\|\|E\|)^{k+1}
\end{align*}

If $M=I-\alpha P$, then $\|\inv M\| = \|\sum_{k=0}^\infty (\alpha
P)^k\| \leq \sum_{k=0}^\infty (\alpha\| P\|)^k = \frac 1
{1-\alpha\|P\|}$, so $(\|\inv M\|\|E\|)^{k+1} \leq
\left(\frac{\|E\|}{1-\alpha\|P\|}\right)^{k+1}$. Since the $P$ is
column stochastic in the PageRank problem, it's 1-norm is 1, so the
relative error in the 1-norm is bounded by
$\left(\frac{\|E\|_1}{1-\alpha}\right)^{k+1}$.

\section{Miscellaneous Observations}

Most of these observations don't seem useful at the moment, but it
seemed like a good idea to keep them in case they prove useful later.

The Sherman Morrison Woodbury formula gives $\inv{(M+E)}=\inv M-\inv
M\inv{(\inv E+\inv M)}\inv M$, which looks similar to the first order
approximation for $(M+E)^{-1}$.

Suppose we have a PageRank iteration $x_{k+1}=\alpha Px_k +
(1-\alpha)v$. Let $e_k=x_k-x_*$, where $x_*$ is the fixed point of the
iteration. Subtracting the equation $x_* = \alpha P x_* + (1-\alpha)v$
from the original iteration equation yields $e_{k+1}=\alpha P e_k$, so
$e_k = (\alpha P)^k e_0$.

Suppose we have a PageRank iteration $x_{k+1}=\alpha Px_k +
(1-\alpha)v$, with initial guess $x_0$. I prove by induction that $x_k
= (\alpha P)^k x_0 + (I-(\alpha P)^{k+1})(I-\alpha P)^{-1}
(1-\alpha)v$. A simple computation shows this is true for $k=0$. If
it's true for $k$, then

\begin{align*}
x_{k+1} &= \alpha Px_k + (1-\alpha)v\\ &= \alpha P \left((\alpha P)^k
x_0 + (I-(\alpha P)^{k+1})(I-\alpha P)^{-1} (1-\alpha)v\right) +
(1-\alpha)v\\ &= (\alpha P)^{k+1}x_0+\left(\alpha P(I-\alpha
P)^{-1}-(\alpha P)^{k+1}(I-\alpha P)^{-1} + I\right)(1-\alpha)v\\ &=
(\alpha P)^{k+1}x_0 + (\alpha P - (\alpha P)^{k+1} + I-\alpha
P)(I-\alpha P)^{-1}(1-\alpha)v\\ &= (\alpha P)^{k+1} x_0 + (I-(\alpha
P)^{k+2})(I-\alpha P)^{-1} (1-\alpha)v
\end{align*}

Suppose we have $x=\alpha Px + (1-\alpha)v$. Note that $\inv{(I-\alpha
  P)} = \sum_{k=0}^\infty (\alpha P)^k = I+\sum_{k=1}^\infty \alpha^k
P^k$. Therefore $x=\inv{(I-\alpha P)}(1-\alpha)v =
(I+\sum_{k=1}^\infty \alpha^k P^k)(1-\alpha)v = v+\sum_{k=1}^\infty
\alpha^k (P^k v -P^{k-1}v) = v+\sum_{k=1}^\infty \alpha^k P^{k-1}
(P-I)v = \left( I+\sum_{k=1}^\infty (\alpha^k P^{k-1}) (P-I)\right)v =
(I+\alpha(I-\alpha P)^{-1})(P-I)v$.

$(M-E)^{-1} = \inv M + \inv M E (M-E)^{-1}$, so $\E{(M-E)^{-1}} = \inv
M + \inv M \E{E(M-E)^{-1}}$. This would be useful if $E$ is drawn from
a probability distribution for which $\E{E(M-E)^{-1}}$ is easier to
compute than $\E{(M-E)^{-1}}$.

\section{Small perturbations and concentration results}

Suppose we write
\[
  M = \bar{M} + Z
\]
where $\bar{M} = E[M]$ is the mean value of $M$ and $Z$ is a
perturbation with $E[Z] = 0$.  If we also define $Y = \bar{M}^{-1} Z$,
then we have
\[
  M^{-1} b = (I - Y + R) \bar{M}^{-1} b
\]
where
\[
  R = (I+Y)^{-1} Y^2.
\]
The relative error in approximating $E[M^{-1} b]$ by $\bar{M}^{-1} b$
can be bounded by
\[
  \frac{\|E[M^{-1} b-\bar{M}^{-1}b]\|}{\|\bar{M}^{-1} b\|} =
  \frac{\|E[R] M^{-1} b\|}{\|\bar{M}^{-1} b\|} \leq \|E[R]\|.
\]

If we can establish that $\|Y\| \leq \gamma < 1$ with probability 1,
then
\[
  \|E[R]\| \leq E[\|R\|] \leq
  E\left[ \frac{\|Y\|^2}{1-\|Y\|} \right] \leq
  \frac{\gamma^2}{1-\gamma}.
\]
The last part of the bound is pessimistic when $\|Y\| \ll \gamma$ most
of the time.  An alternative bound is to use integration by parts
to obtain
\[
  E\left[ \frac{\|Y\|^2}{1-\|Y\|} \right] =
  \int_0^\gamma \frac{2t-t^2}{(1-t)^2} P\{ \|Y\| \geq t \} \, dt,
\]
then bound the tail probability $P\{ \|Y\| \geq t \}$.

A weak bound on the tail probability for $\|Y\|$ (e.g.~via Markov's
inequality applied directly) does not provide a very interesting
result.  If the entries of $Y$ are independent random variables, a
more promising approach is to consider {\em concentration of measure}
results.  In the case of PageRank, we need to take into account the
fact that even if the edge weights are independent random variables,
the entries of the matrix $P = AD^{-1}$ become dependent via the
normalization.  We probably also need to add self-loops to guarantee
non-singularity.

% Demo how to use BibTeX
\nocite{Constantine:2009:Random}
\bibliographystyle{siam}
\bibliography{refs}
 
\end{document}
